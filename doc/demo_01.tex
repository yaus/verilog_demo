\documentclass[12pt,a4paper]{article}

\usepackage[a4paper,margin=1in]{geometry}
\usepackage[utf8]{inputenc}
\usepackage{gillius2}
\usepackage{calc}


\usepackage{hyperref}
\hypersetup{pdftex,colorlinks=true,allcolors=blue}
\usepackage{bookmark}
\bookmarksetup{
  numbered, 
  open,
}
\usepackage{hypcap}

\usepackage[style=numeric]{biblatex}
\usepackage[english]{babel}
\usepackage{csquotes}
\addbibresource{demo\_01.bib}

\setlength{\parindent}{0pt}

\usepackage{fontspec}
\setmainfont{Gillius ADF No2}

\usepackage{xcolor}
\usepackage{listings}
\definecolor{vgreen}{RGB}{104,180,104}
\definecolor{vblue}{RGB}{49,49,255}
\definecolor{vorange}{RGB}{255,143,102}

\lstdefinestyle{verilog-style}
{
    language=Verilog,
    basicstyle=\small\ttfamily,
    keywordstyle=\color{vblue},
    identifierstyle=\color{black},
    commentstyle=\color{vgreen},
    numbers=left,
    numberstyle=\small\color{green},
    numbersep=12pt,
    tabsize=4,
    moredelim=*[s][\colorIndex]{[}{]},
    literate=*{:}{:}1,
    framexleftmargin=30pt
}

\usepackage{titlesec}
 
\titleformat{\section} 
	{\normalfont\Large\bfseries}{\makebox[30pt][l]{\thesection}}{0pt}{} 
\titleformat{\subsection} 
	{\normalfont\large\bfseries}{\makebox[30pt][l]{\thesubsection}}{0pt}{}

\renewcommand{\lstlistingname}{Source Code}% Listing -> Algorithm
\renewcommand{\lstlistlistingname}{List of \lstlistingname s}% List of Listings -> List of Algorithms

\makeatletter
\newcommand*\@lbracket{[}
\newcommand*\@rbracket{]}
\newcommand*\@colon{:}
\newcommand*\colorIndex{%
    \edef\@temp{\the\lst@token}%
    \ifx\@temp\@lbracket \color{black}%
    \else\ifx\@temp\@rbracket \color{black}%
    \else\ifx\@temp\@colon \color{black}%
    \else \color{vorange}%
    \fi\fi\fi
}
\makeatother

\title{Demo 01: Verilog \& Icarus Verilog\vspace{-1ex}}
\author{Eddy Yau\vspace{-2ex}}
\date{\today}

\begin{document}
\maketitle 
\section{Introduction}
\subsection{Verilog}
Verilog and VHDL are the most commonly used \textbf{HDL}(Hardware Description Language) in the world. 
On the other hand, high-level synthesis will utilize SystemC instead. 
Comparing with verilog and VHDL, verilog is more easily to study, since verilog is similar to C.
\newline
Here is some example code for a NAND gate in verilog:

\lstset{tabsize=2,breaklines=true,numbers=left,basicstyle=footnotesize,xleftmargin=30pt}


\begin{lstlisting}[style={verilog-style},frame=single,caption=NAND.v]{NAND gate}
module NAND {
	input  A,
	input  B,
	output Z
};
// Assign nand-gate logic to Z
assign Z=~A&B;
endmodule
\end{lstlisting}

\smallskip
Declaration for a module:
\begin{lstlisting}[style={verilog-style},frame=single,firstnumber=1]
module NAND {
\end{lstlisting}

\smallskip
Pin list of the module
\begin{lstlisting}[style={verilog-style},frame=single,firstnumber=2]
	input  A,
	input  B,
	output Z
\end{lstlisting}

\smallskip
Assign combinational logic to output
\begin{lstlisting}[style={verilog-style},frame=single,firstnumber=7]
assign Z=~A&B;
\end{lstlisting}

\smallskip
End of module declaration
\begin{lstlisting}[style={verilog-style},frame=single,firstnumber=8]
endmodule
\end{lstlisting}
\subsection{Icarus Verilog}
Icarus Verilog is a Verilog simulation and synthesis tool.It operates as a compiler, compiling source code written in Verilog (IEEE-1364) into some target format.\cite{iverilogwebsite} It is free to use.

\medskip
\textbf{Iverilog} is the compiler to synthesis the verilog source code to vvp assembly.
\textbf{Vvp} is the tool to execute the vvp assembly.

\section{Simulation}
In this demo, we will create a ``Hello World'' program executing by Icarus Verilog. For this demo, we are using Ubuntu 17.10 dist.
\subsection{Install}
First, we install the Icarus Verilog. Open a terminal and install Icarus Verilog via apt
\begin{lstlisting}[language=bash,basicstyle=\small\ttfamily,framexleftmargin=30pt,frame=single,numbers=none]
$> sudo apt install iverilog
\end{lstlisting}
\subsection{Create verilog source code}
Create a verilog file as below:
\begin{lstlisting}[style={verilog-style},frame=single,firstnumber=1,caption=hello\_world.v]
module hello_world;
initial begin
    display(Hello World!);
    finish;
end
endmodule
\end{lstlisting}
\subsection{Compile \& Execute}
\begin{lstlisting}[language=bash,basicstyle=\small\ttfamily,framexleftmargin=30pt,frame=single,numbers=none]
$> iverilog -o hello_world.vvp hello_world.v
$> vvp hello_world.vvp
Hello World!
\end{lstlisting}
\printbibliography

\end{document}

