\documentclass[12pt,a4paper]{article}

\usepackage[a4paper,margin=1in]{geometry}
\usepackage[utf8]{inputenc}
\usepackage{gillius2}

\usepackage{fontspec}
\setmainfont{Gillius ADF No2}

\usepackage{xcolor}
\usepackage{listings}
\definecolor{vgreen}{RGB}{104,180,104}
\definecolor{vblue}{RGB}{49,49,255}
\definecolor{vorange}{RGB}{255,143,102}

\lstdefinestyle{verilog-style}
{
    language=Verilog,
    basicstyle=\small\ttfamily,
    keywordstyle=\color{vblue},
    identifierstyle=\color{black},
    commentstyle=\color{vgreen},
    numbers=left,
    numberstyle=\small\color{green},
    numbersep=12pt,
    tabsize=8,
    moredelim=*[s][\colorIndex]{[}{]},
    literate=*{:}{:}1,
    framexleftmargin=30pt
}

\makeatletter
\newcommand*\@lbracket{[}
\newcommand*\@rbracket{]}
\newcommand*\@colon{:}
\newcommand*\colorIndex{%
    \edef\@temp{\the\lst@token}%
    \ifx\@temp\@lbracket \color{black}%
    \else\ifx\@temp\@rbracket \color{black}%
    \else\ifx\@temp\@colon \color{black}%
    \else \color{vorange}%
    \fi\fi\fi
}
\makeatother

\title{Demo 01: Icarus Verilog\vspace{-1ex}}
\author{Eddy Yau\vspace{-2ex}}
\date{\today}

\begin{document}
\maketitle
\section*{Verilog}
Verilog and VHDL are the most commonly used \textbf{HDL}(Hardware Description Language) in the world. 
On the other hand, high-level synthesis will utilize SystemC instead. 
Comparing with verilog and VHDL, verilog is more easily to study, since verilog is similar to C.
\newline
Here is some example code for a NAND gate in verilog:

\lstset{tabsize=2,breaklines=true,numbers=left,basicstyle=footnotesize,xleftmargin=30pt}

\begin{lstlisting}[style={verilog-style},frame=single]{NADN gate}
module NAND {
	input  A,
	input  B,
	output Z
};
// Assign nand-gate logic to Z
assign Z=~A&B;
endmodule
\end{lstlisting}



\end{document}
